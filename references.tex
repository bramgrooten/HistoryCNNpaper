

\begin{thebibliography}{}



\bibitem{effps} R. Mohan and A. Valada. (2020). \textit{EfficientPS: Efficient Panoptic Segmentation}. \url{http://panoptic.cs.uni-freiburg.de/}

\bibitem{dl-book} I. Goodfellow, Y. Bengio and A. Courville. (2016). \textit{Deep Learning}. MIT Press. \url{http://www.deeplearningbook.org/}

%% soms hoofdstuk 1, soms hfst 9: fix dit in de tekst 


\bibitem{first} \url{http://historyofinformation.com/detail.php?entryid=782}

\bibitem{mcc} McCulloch, W. S., \& Pitts, W. (1943). A logical calculus of the ideas immanent in nervous activity. The bulletin of mathematical biophysics, 5(4), 115-133.

\bibitem{percep} Rosenblatt, F. (1958). \textit{The perceptron: A probabilistic model for information storage and organization in the brain}. Psychological Review, 65(6), 386–408. https://doi.org/10.1037/h0042519

\bibitem{med} https://towardsdatascience.com/mcculloch-pitts-model-5fdf65ac5dd1

\bibitem{fullpic} \url{https://www.linkedin.com/pulse/deep-learning-nutshell-part-1-ankit-agarwal/}

\bibitem{dl-lecun} LeCun, Y., Bengio, Y. & Hinton, G. \textit{Deep learning}. Nature 521, 436–444 (2015). \url{https://doi.org/10.1038/nature14539}

\bibitem{jap} K. Fukushlma, Neural network model for a mechamsm
of pattern recognmon unaffected by shift m
posltlon--neocogmtron--, Trans Inst electromcs commun
Enors Japan 62-A, 658-665 (1979) (In Japanese)

\bibitem{neocog} K. Fukushima, \textit{Neocognitron a self-organizing neural
network model for a mechanism of pattern recognition
unaffected by shift in position}, Biological Cybernetics 36,
193-202 (1980)

\bibitem{sch} Schmidhuber, Jürgen (2015). "Deep Learning". Scholarpedia. \url{http://www.scholarpedia.org/article/Deep_Learning}

\bibitem{about-fuku} "Kunihiko Fukushima". The Franklin Institute. \url{https://www.fi.edu/laureates/kunihiko-fukushima}

\bibitem{fuzzy} \url{http://personalpage.flsi.or.jp/fukushima/index-e.html}

\bibitem{history} \url{https://www.import.io/post/history-of-deep-learning/}

\bibitem{83} Fukushima, K. (2019). Efficient IntVec: High recognition rate with reduced computational cost. Neural Networks, 119, 323-331.

\bibitem{hubel} Hubel, D.H., Wiesel, T.N. \textit{Receptive fields, binocular interaction
and functional architecture in cat's visual cortex}. J. Physiol.
(London) 160, 106-154 (1962)

\bibitem{wiesel} Hubel, D.H., Wiesel, T.N. \textit{Receptive fields and functional architecture in two nonstriate visual area (18 and 19) of the cat}. J.
Neurophysiol. 28, 229-289 (1965)



















\end{thebibliography}