
\section{What are the applications of ConvNets?}
\label{sec:app}

Algorithms based on ConvNets have become dominant in computer vision tasks in the last decade \cite{radiol}. Tasks include \textit{image classification}, which requires the algorithms to detect which items are present in the image, as well as \textit{object detection}, which means that an algorithm needs to specify the location of the objects in the image. Both tasks can be gather under the term \textit{object recognition} \cite{ilsvrc}.\\

The recognition algorithms with ConvNets experienced one of their first breakthroughs when Yann LeCun applied them for optical character recognition (OCR) in 1989 \cite{ocr}. Out of this innovation emerged bank check reading systems, that could accurately recognize handwriting. By the late 90's these systems were already used for about 10\% of all checks in the US \cite{convnet}. Nowadays handwriting recognition systems are used to digitize historical documents \cite{histdoc}.\\

Image classification is a useful application in Google image search for instance. In 2012 results in this field were greatly improved on the ImageNet visual recognition  ompetition. The moment was called a ``turning-point for large-scale object recognition'' \cite{ilsvrc}. Autonomous cars can be made safer if they have an impeccable object recognition system on board. Research in this area is quickly developing, as shown in the image on the title page of this report \cite{effps}.\\

Another application of ConvNets is facial recognition, which has been worked on since at least 1997 \cite{face}. After systems were able to detect faces, research tried to distinguish between different facial expressions \cite{express} and the next step was detecting emotions. Recently, attempts are being made to make the facial recognition systems invariant on the amount of illumination \cite{illu}. The technology is being used for many kinds of security systems, like unlocking a smartphone.\\

ConvNets are very useful in the analysis of medical images. They have been applied in research on brain activity \cite{brain}, as well as the detection of certain types of cancer \cite{cancer}. These are good examples of applications that are highly beneficial to humankind. There are however other applications which could be risky. Next to image recognition, ConvNets can also be used to generate (fake) images, such as highly realistic portrets \cite{fake}.\\

All the applications decribed so far have been in the field of computer vision, where ConvNets are often used. There are ConvNet applications outside this area as well. Examples include detecting extreme weather in climate datasets \cite{climate},
speech recognition \cite{speech}, 
or even trying to understand spoken or written sentences with natural language processing (NLP) \cite{nlp}. This last paper used networks with up to 29 convolutional layers, and therefore called them `Very Deep Convolutional Networks'. NLP is an important component of automated customer service systems.\\


